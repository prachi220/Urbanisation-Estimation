Remotely sensed imagery has been extensively used to identify landcover changes occurring in the urban areas and its peripheral rural areas. A lot of efforts have been put in performing spatial analysis on satellite imagery like LANDSAT and Sentinel. There has been quite a development in the methods used to estimate the landcover areas and hence urbanisation. New indices have been devised to indicate different aspects of a human settlement, like Normalized Difference Vegetation Index (NDVI), Normalized Difference Built Index (NDBI), Normalized Difference Water Index (NDWI), etc., which can typically be calculated using only four bands, namely red, green, near infrared (NIR) and short-wave infrared (SWIR). Most of the initial work involving satellite data used these parameters to give estimates on the settelment areas, and hence determine urbanisation. Macarof et al.\cite{ndvindbiheatisland} have used NDBI and NDVI as indicators for Surface Urban Heat Island Effect. Zha et al.\cite{ndbiurbanisation} have used NDBI in automatically mapping urban areas from TM imagery. 

Further, these estimates have been used to classify the land into vegetation land, barren land and a builtup region. Szabo et al.\cite{ndvi1} worked on specific features of NDVI, NDWI and NDBI as reflected in land cover categories. These spectral indices were used to determine land cover types: water body (W); plough land (PL); forest (F); vineyard (V); grassland (GL) and built-up areas (BU) using Landsat-7 ETM+ data. They identified that the categories of BU, GL and F could be calculated using NDVI values, but the other land cover types differed significantly. Further, they investigated the ranges of these spectral indices from the aspect of land cover types. Similarly, Faridatul et al.\cite{novelspectralindices} worked on estimating urban landcovers based on Novel Spectral Indices which were defined as MNDBI, TCWVI, ShDI. These indices were calculated using the basic bands and the initial indices of NDVI, NDBI and NDWI. They then devised an approach to classify four major urban land types: impervious, bare land, vegetation, and water. 

Further, there have been approaches to automate the process of classfication by building models over the existing bands and indices as the input parameters. And it has been one of the most important applications developed using Earth observation satellites. Phiri et al.\cite{classificationreview}, in their paper, review the developments in landcover classification methods for Landsat images in the last four decades. This review suggests that initial approached involved visual analyses, followed by unsupervised and supervised pixel-based classification methods using maximum likelihood, K-means, etc. In the year 2015, Erzhu Li et al.\cite{ndvi2} gave an automatic approach for urban land cover classification from Landsat 8 OLI data. They used object oriented methods, instead of pixel based classification and introduced new parameters over the existing ones like NDVI, NDWI, the modified normalized difference water index (MNDWI), the soil adjusted vegetation index (SAVI) etc., which then used a non-linear SVM to train over these data points.

Goldblatt et al.\cite{goldblatt2016detecting} gave some revolutionary work on remote sensing to identify the boundaries of urban areas using pixel wise classification. They constructed and validated a large-scale and comprehensive dataset of 21,030 points (4682 marked as builtup(BU) and 16,348 marked as non-builtup) designed for mapping urban areas in India, and then demonstrated its applicability for mapping urban areas in India. They also give a classifier to perform the task of classification in Google Earth Engine (GEE) and evaluate its spatial generalizability, using a spatial k-fold cross-validation procedure. They utilize the full spectral imagery available in Landsat, and NDVI and NDBI indices(Normalized difference Built-up Index) indices, and incorporate the agro-climatic zones found in the large majority of developing countries. These agro-climatic zonesare geographical regions characterized by relatively homogenous environmental-physical characteristics, such as soil type, rainfall, temperature, and water resources. While this approach has proven to give promising results, upon observation we identify that the classification is deeply affected by the inherent noise present in the input satellite data, discussed in section ~\ref{sub:cloudcover}. Also, we further observe that since the classification approach has its own limitations, it mightbe preferable to use the results for a temporal analysis to record for the changes, instead of a cross-sectional analysis.

Based on similar lines, Goldblatt et al.\cite{vietnam} did some work using satellite data for urban classification in Vietnam. They designed a tool to map built-up landcover (LC)/ landuse (LU) regions in Ho Chi Minh City, Vietnam, using the publicly available satellite data and a cloud-based computational platform. They then mapped the temporal changes in the extent of built-up land cover in the province over the period 2000 to 2015. They used GDLA administrative cadastral data (polygons) and 15,945 hand-labeled examples (points) as reference data for supervised pixel-based image classification into “not-built-up” land cover, “residential” and “non-residential” land use. This analysis also gives a reasonable estimate on the distribution of the urban regions, and not urbanisation. Also, detailed GDLA data being used here is available only for Vietnam, and hence cannot be extended to India.

We therefore attempt to give an approach to estimate urbanisation, by providing an end to end pipeline, each component of which addresses one or the other shortcomings of the work that has been done so far and closely aligns with the aim and the methodology of this work.