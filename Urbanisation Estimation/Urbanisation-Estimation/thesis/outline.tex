The aim of our work is to establish methods and options for supervised training and classification of built up (BU) and non-built up (NBU) pixels in various districts of India, for years 2014 to 2018. We then establish methods to draw conclusions on temporal changes in the land-use pattern for different districts. These changes account for the urbanisation that has happened in the concerned region in the given duration of time. And hence as discussed before, to get a better estimate of the urbanisation, it is important to account for the extent of change as well as the distribution pattern of the change over the region. Then we generate a groundtruth dataset to validate our temporal analysis and generate a set of 9 parameters to represent the urbanisation index. Throughout our analysis, we have used Gurgaon and Jaipur to show the results, unless mentioned otherwise. The outline of the pipeline is as follows:

\begin{enumerate}
	\item \hyperref[method:obtainimages]{{\bf Obtaining BU/NBU classified images:}} We train a classifier on an available groundtruth dataset and run it on the Landsat-8 images of years 2014 to 2018. We identify that performing classification on the raw images leads to noisy and unsatisfactory results, which can be owed to the variable cloud cover index spatially and temporally. We then apply some correction methods to improve the results.

	\item \hyperref[method:urbanisationindices]{{\bf Generating urbanisation indices using the classified images:}} We generate a total of 9 parameters to account for urbanisation, 3 of which represent the extent (amount) of change, while the rest 6 parameters represent the patterns in change during the given period.\\

		\begin{itemize}
			\item \hyperref[method:extentofchange]{{\bf Extent of change using Temporal Analysis:}} Since the Landsat images available have their own limitations and that the classifier trained mightnot be the most accurate, we  understand that the absolute pixel wise classification results maynot align with the ground truth. To address this, we apply some spatial smoothing techniques on the classification results to address the noisy pixels, and then do a temporal analysis over the span of 5 years, to obtain the said 3 parameters that can represent the extent of change for urbanisation. Since at various steps, we need to decide on threshold values, it is important to justify them by validating them against the ground truth. We therefore generate this dataset and perform the validation.

			\item \hyperref[method:patternofchange]{{\bf Patterns in change using Blob Detection:}} Urbanisation doesn't just account for the percentage change across a period of time, but also the patterns of change in the distribution. Urbanisation might be expansion around the existing urban clusters, or emergence of new settlement areas altogether. The overall distribution of the population in a district (for example, monocentric vs polycentric) also represent urbanisation. We therefore generate a few parameters using the method of blob detection.
		\end{itemize}
\end{enumerate}