It is essential to note that our analysis does not merely give parameters for urbanisation, but is also able to identify the main urban clusters in Indian districts. This knowledge can help the government streamline the focus of their policies and narrow it down to either these urban clusters or remote rural settlements. The validation set that has been generated for Gurgaon and Jaipur can find many applications in different fields, and if extended further to other districts can then be useful for training and validating various models and analyses.

Finally we have tried to understand urbanisation based on the presence of buildings and infrastructure in a region. While this is a very significant factor in deciding whether the region in concern is urbanised or not, we understand that there are many other socio-economic factors that play a role in deciding if the population in an area is urbanised. To be able to clearly define the urbanisation process, it is important to understand the significance of all these factors and the correlation between them. Since there is always a scope of improving on the way we define urbanisation, we understand that there is a long way to go and a lot of other factors that need to be taken into consideration to obtain a thorough and vivd picture of changing cities.